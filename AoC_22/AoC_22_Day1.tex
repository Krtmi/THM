\addcontentsline{toc}{subsection}{AoC 22, Day 1: Someone's coming to town!}
\subsection*{AoC22, Day 1: Someone's coming to town!}
This day's thematic revolves around Security Frameworks: 
\subsubsection*{NIST Cybersecurity Framework}
	Five essential functions:
	\begin{itemize}
	\item Identify
	\item Protect
	\item Detect
	\item Respond
	\item Recover
	\end{itemize}
	This way, prioritise cybersec investments and engage in continuous improvement towards target cybersec profile. 
	
\subsubsection*{ISO 27000 Series}
ISO as International Organization of Standardization. Especially relevant are ISO 27001 and ISO 27002

\subsubsection*{MITRE ATT\&CK Framework}
Knowledge base of \textbf{Tactics, Techniques and Procedures (TTP)} to understand behaviours. methods, tools and strategies established for an attack to identify adversary plans of attack. 
\subsubsection*{Cyber Kill Chain}
Describes the structure of an attack: 
\begin{enumerate}
\item Target identification
\item Decision and order to attack the target
\item Target destruction
\end{enumerate}

This further establishes 7 stages: 
\begin{enumerate}
\item Recon
\item Weaponization
\item Delivery
\item Exploitation
\item Installation
\item Command \& Control
\item Actions on Objectives
\end{enumerate}

\subsubsection*{Unified Kill Chain}
Unification of \textbf{MITRE ATT\&CK and Cyber Kill Chain} frameworks.\\
Published in 2017, reviewed in 2022 by Paul Pols.\\
Model to defend against cyber attacks from adversary's perspective offering sec teams blueprint for analyising and comparing threat intel.\\
3 cycles, 18 phases in total based on TTPs:\\

\textbf{CYCLE 1: IN}\\
Goal: Gain access to the system.
Critical steps: 
\begin{enumerate}
\item Reconnaissance: Research on the target using public information
\item Weaponisation: Set up the structure for Command and Control (C2)
\item Delivery: Deliver payloads to target in form of e.g phishing and supply chain attacks
\item Social Engineering: Trick the target into performing untrusted and unsafe action against received payload
\item Exploitation: Use an eventual vulnerability to trigger the payload
\item Persistence: Leave a fallback presence to be able to reaccess the target
\item Defence Evasion: Remain anonymous, avoiding defence mechanisms, deleting evidence of their presence
\item Command \& Control: Establish a communicaation channel between attacker and compromised structure using step 2
\end{enumerate}

\textbf{CYCLE 2: THROUGH}
Goal: gaining more access and privileges to assets within the network.\\
Do: 
\begin{enumerate}
\item Pivoting: Use persistence system as attack launchpad for other systems un the network
\item Discovery: Gather information about compromised system or discover further vlnerabilities
\item Privilege Escalation: Seek higher privileges exploiting vulnerabilites or misconfigurations
\item Execution: Download and execute malicious code\
\item Credential Access: Seek stored credentials as part of the attack
\item Lateral Movement: Move around different systems using above credentials
\end{enumerate}

\textbf{CYCLE 3: OUT}\\
Goal: Cause as much damage as possible and escape undetected. 
Steps:
\begin{enumerate}
\item Colection: Aggregate all information
\item Exfiltration: Get the information out of the network
\item Impact: Use acquired privileges to manipulate, interrupt and sabotage
\item Objectives: Define and understand possible further objectives of the attackers
\end{enumerate}

\textbf{Question 1:}\\
Who is the adversary that attacked Santa's network this year?\\
\textbf{Answer 1:}\\
The Bandit Yeti\\

\textbf{Question 2:}\\
What's the flag that they left behind?\\
\textbf{Answer 2:}\\
THM\{IT'S A Y3T1 CHR1\$TMA\$\}
