\section*{General Attacking Steps}
\addcontentsline{toc}{section}{General Attacking Steps}
\begin{enumerate}
\item Enumeration via a Port Scan with Nmap.\\
More on Nmap in the writeup for the \nameref{Nmap} exercises.
\end{enumerate}
When auditing a website checking for vulnerabilites:\\
Check the \nameref{Methodology_WebSite_Audit_Vuln} or the \nameref{Stepwise_WriteUp_Web_Hacking} enumeration for an actual stepwise CTF.
\subsection*{Nmap switches}
\addcontentsline{toc}{subsection}{Nmap switch summary}
\begin{tabular}{|r|l|}
\hline
\textbf{Host discovery:} & \\
\hline
\cd{-Pn} & Treat all hosts as online\\
\cd{-p a-b} & Scan ports a to b\\
\cd{-p-} & Scan all ports\\
\hline
\raggedleft{\textbf{Scans:}}& \\
\hline
\textbf{TCP Scans:} & \\
\cd{-sS} & SYN Scan\\
\cd{-sT} & Connect() Scan\\
\cd{-sA} & ACK Scan\\
\cd{-sN} & Null Scan\\
\cd{-sF} & FIN Scan\\
\cd{-sX} & Xmas Scan\\
\cd{-sO} & IP Protocol Scan\\
\hline
\textbf{Other scans:} &\\
\cd{-sC} & Default script scan\\
\cd{-A} & Aggressive scan: \cd{-O -sV-sC --traceroute}\\ 
\hline
\textbf{Service detection:} & \\
\cd{-sV} & Version info\\
\hline
\textbf{OS Detection:} & \\
\cd{-O} & OS detection\\
\hline
\end{tabular}
\subsection*{Metasploit Reverse Shells}
See \nameref{Reverse_Metasploit_Steps}