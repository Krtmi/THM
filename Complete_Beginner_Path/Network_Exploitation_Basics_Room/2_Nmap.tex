\addcontentsline{toc}{subsubsection}{Nmap}
\subsubsection*{Nmap}
\label{Nmap}
\T{Introduction}{
In order to assess the IP we are attacking we want to learn which services are running where on the target.\\
In order do this, we do a port scanning, since any connection needs a port to establish, e.g to discern which tab loads which website the connecting computer uses different ports.\\ 
Similarly, one needs to use different ports to run HTTP and HTTPS versions of the site.\\
Network connections are established between a high numbered port in the connecting computer, assigned at random, and a listening port on the server.\\
Every computer has a total of 65535 available ports, but some of them are registered as standard ports:
\begin{itemize}
\item HTTP: port 80
\item HTTPS: port 443
\item Windows NETBIOS: 139
\item SMB: 445
\end{itemize}
In many CTF environments the standard port assignment is altered, though, making enumeration even more important.\\
Ports can be open, closed or filtered, ususally by a firewall\\

To know which ports are open we need to perform an enumerating port scan as the first step of every attack, which we perform with \cd{nmap}, a tool to perform all kinds of analysis.\\
Nmap is the best industry standard due to its functionality, high power and the possibility to even perform the exploit itself.
\QA{
What networking constructs are used to direct traffic to the right application on a server?
}{
Ports
}
\QA{
How many of these are available on any network-enabled computer?
}{
65535
}
\QA{
[Research] How many of these are considered "well-known"? (These are the "standard" numbers mentioned in the task)
}{
The "well-known" ports are ranging 0 to 1023, hence are \textbf{1024}
}
}


\T{Nmap Switches}{
This task is about the different possibilities an Nmap scan allows for:\\
after taking a look at \cd{nmap -h} or \cd{man nmap} we can start answering the questions.
\QA{
What is the first switch listed in the help menu for a 'Syn Scan' (more on this later!)?
}{
Running \cd{nmap -h | grep SYN} we see the switches \\
\cd{-sS/sT/sA/sW/sM: TCP SYN/Connect()/ACK/Window/Maimon scans} and hence the answer is \textbf{\cd{-sS}}
}
\QA{
Which switch would you use for a "UDP scan"?
}{
We can directly see that \textbf{\cd{-sU}} is the switch for UDP scan
}
\QA{
If you wanted to detect which operating system the target is running on, which switch would you use?
}{
\cd{-O}
}
\QA{
Nmap provides a switch to detect the version of the services running on the target. What is this switch?
}{
With \cd{nmap -h |grep version} we see the switch:\\
\cd{-sV: Probe open ports to determine service/version info}, hence \textbf{\cd{-sV}} is the answer.
}
\QA{
The default output provided by nmap often does not provide enough information for a pentester. How would you increase the verbosity?
}{
\cd{-v}
}
\QA{
Verbosity level one is good, but verbosity level two is better! How would you set the verbosity level to two?
(Note: it's highly advisable to always use at least this option)
}{
As explained in the \cd{-v} switch, \textbf{\cd{-vv}} increases verbosity further
}
We should always save the output of our scans -- this means that we only need to run the scan once (reducing network traffic and thus chance of detection), and gives us a reference to use when writing reports for clients.
\QA{
What switch would you use to save the nmap results in three major formats?
}{
\cd{\textbf{-oA} <basename>: Output in the three major formats at once}
}
\QA{
What switch would you use to save the nmap results in a "normal" format?
}{
From the first \cd{OUTPUT} switch we see:\\
\cd{-oN/-oX/-oS/-oG <file>: Output scan in normal, XML, s|<rIpt kIddi3,and Grepable format, respectively, to the given filename.}, hence \cd{-sN} is the normal output we seek.
}
\QA{
A very useful output format: how would you save results in a "grepable" format?
}{
From the response above we further see that the grepable format requires the \textbf{\cd{-oG}} switch
}
Sometimes the results we're getting just aren't enough. If we don't care about how loud we are, we can enable "aggressive" mode. This is a shorthand switch that activates service detection, operating system detection, a traceroute and common script scanning.
\QA{
How would you activate this setting?
}{
In the miscellaneous section on the manual we find the switch \cd{-A (Aggressive scan options)}, which is a mix of OS detection, version scanning, script scanning and traceroute, hence amounting to running \\
\cd{nmap <target> -O -sV -sC --traceroute}\\
\textbf{This option is not to be used against targets without permission!}\\
It also does not enable timing or verbosity options.
}
Nmap offers five levels of "timing" template. These are essentially used to increase the speed your scan runs at. Be careful though: higher speeds are noisier, and can incur errors!
\QA{
How would you set the timing template to level 5?
}{
\cd{-T5}
}
We can also choose which port(s) to scan.
\QA{
How would you tell nmap to only scan port 80?
}{
\cd{ -p 80}
}
\QA{
How would you tell nmap to scan ports 1000-1500?
}{
\cd{-p 1000-1500}
}
A very useful option that should not be ignored:
\QA{
How would you tell nmap to scan all ports?
}{
\cd{-p-}
}
\QA{
How would you activate a script from the nmap scripting library (lots more on this later!)?
}{
\cd{--script}
}
\QA{
How would you activate all of the scripts in the "vuln" category?
}{
Since the category of scripts to use is set by \cd{--script=<name>}, we simply set\\
\cd{--script=vuln}
}


\T{Scan Types: Overview}{
As stated in the Nmap manual, the basic part of the port scan switches is \cd{\textbf{-s}C}, where C is "a prominent character in the scan name, usually the first". This way, we have:
\begin{itemize}
\item TCP Connect Scans: \cd{-sT}
\item SYN "Half-open" Scans: \cd{-sS}
\item UDP Scans: \cd{-sU}
\end{itemize}
And further, less common scan types to avoid being detected by firewalls and other detection and intrusion prevention system.
\begin{itemize}
\item TCP Null Scans: \cd{-sN}
\item TCP FIN Scans: \cd{-sF}
\item TCP Xmas Scans: \cd{-sX}
\end{itemize}
This last scan type, the Xmas scan, has the FIN, PSH and URG flags, hence "lighting it up like a christmas tree"
}
This Task being purely introductory, the exercise is only a checkbox on the reading of this text. 
}
\T{Scan Types: TCP Connect Scans}{
In this task we'll deal with TCP Connect Scans (\cd{-sT}), which deals with the TCP three-way handshake.\\
As a recapitulation, the client send a TCP request with the SYN (synchronization) flag set. This gets another TCP response with the SYN / ACK flag.\\
In the end, the TCP handshake gets completed by sending the last TCP request with the ACK (acknowledgement) flag set.\\

This handshake procedure is useful for the Nmap scan, as RFC 9293 states: \\
\textit{"... If the connection does not exist (CLOSED), then a reset is sent in response to any incoming segment except another reset. A SYN segment that does not match an existing connection is rejected by this means."}\\
This means that any closed port will answer with a RST (reset) flag, hence identifying itself as such.\\
Whenever this request is sent to an open port, the target will continue performing the handshake, identifying a possible connection to that port.\\
Nevertheless, Nmap completes the handshake, which slows down a bit the procedure.\\
The third possibility, an open port behind a firewall, doesn't get identified as such since most firewalls simply drop such TCP SYN requests. Still, some firewalls are configured to send an RST packet, and it is not difficult to configure it, e.g IPtables, to do so in order to make the identification more difficult: \\
\cd{iptables -I INPUT -p tcp --dport <port> -j REJECT --reject-with tcp-reset}
The questions are a reading comprehension on the previous text:
\QA{
Which RFC defines the appropriate behaviour for the TCP protocol?
}{
RFC 9293
}
\QA{
If a port is closed, which flag should the server send back to indicate this?
}{
RST
}
}


\T{Scan Types: SYN Scan}{
This task deals with SYN Scans (\cd{-sS}), also called "half-open" or "stealth" scans.\\
It relies on the TCP handshake as well as the TCP Connect Scan, but without completing it and sending a RST packet instead of the last ACK packet to abort the connection and prevent the server from repeatedly trying to re-establish the connection.\\
This has a myriad of advantages:\\
\begin{itemize}
\item It can be used to bypass older Intrusion Detection systems, which look for a full three way handshake.\\
This characterizes this scan as stealth scan. 
\item They are not logged by applications listening, as usually the only logged connections are the established ones.
\item It is faster than a TCP scan as it does not need to fully connect and disconnect to every port
\end{itemize}
But it also comes with some disadvantages:
\begin{itemize}
\item It requires \cd{sudo} permissions to work correctly since it needs the permissions to create raw packets.
\item The repeated interrupting of the handshake may bring down unstable systems, a big downside when analysing a critical system for a client.
\end{itemize}
Still, the pros outweight the cons, hence Nmap runs a SYN \cd{-sS} Scan per default when running with \cd{sudo} permissions. Else the default ist the TCP Scan of the previous task.\\
The SYN scan works exactly the same as the TCP Scan when attempting to identify closed and filtered ports: a closed port will respond with a RST TCP packet, a filtered one drops the TCP SYN packet from the sending client (our computer) or spoofs it with a TCP RST to hide it.\\
Hence, the only difference relies in the handling of open ports.
}


\T{Scan Types: UDP Scans}{
UDP connections being stateless, they don't have a handshake  to confirm the connection by. They send their packets and hope that they reach the target, having its inherent advantages but being very difficult and slower to scan.\\
Nonetheless, we can perform a UDP Scan on Nmap with the switch \cd{-sU}, which will only responds on closed ports.\\
As opposed to open or filtered ports, which commonly accept the UDP packets without response and are marked as \cd{open | filtered} after the second attempt without response or as \cd{open} in the unlikely event an UDP response is sent back, closed ports respond with an ICMP (ping) packet declaring the port as unreachable.\\
The number of attempts and difficulty in the definition make this scan very slow ($\sim$20' for 1000 ports), so one usually runs this only on a limited number of ports via\\
\cd{nmap -sU --top-ports <number> <target>}\\
Instead of sending the common raw UDP packets, for the ports occupied by well-known services it will send a protocol-specific payload more likely to elicit a response to draw the right conclusion from.\\
Questions on this Task are still reading comprehension:
\QA{
If a UDP port doesn't respond to an Nmap scan, what will it be marked as?
}{
\cd{open | filtered}
}
\QA{
%When a UDP port is closed, by convention the target should send back a "port unreachable" message. Which protocol would it use to do so?
}{
ICMP
}
}


\T{Scan Types: NULL, FIN and Xmas}{
This task presents the less commonly used NULL, FIN and Xmas scans, which are even stealthier than the previous SYN scan.\\
They all identify closed ports as those sending a RST response, non-responding ports as open|filtered and ports sending an ICMP unreachable packet as filtered.\\

The NULL scan \cd{-sN} sends a TCP request with no flag at all, eliciting a RST response if the port is closed.\\

The FIN scans \cd{-sF} works similarly as the NULL scan but sending a FIN flag, commonly used to close a connection. \\

The Xmas scan \cd{-sX} sends a malformed TCP package with PSH, URG and FIN flags set. The name comes from the non-consecutive placement from the set flags, which give the analysed package the appearence of a Christmas tree when viewed as a packet capture in Wireshark.\\

This should work according to RFC 793, but isn't always the case: Microsoft Windows and Cisco network devices respond with a RST packet to all malformed packets sent to them, hence marking all ports as closed under a Nmap scan.\\

Nonetheless, these scans are still useful for firewall evasion, as many of them are configured to drop incoming SYN TCP packets to blocked ports. Sending requests without said flag these scans manage to avoid this Intrusion Detection system.\\
Still, most modern IDS have already caught up and implement countermeasures against these scans.\\

\QA{
Which of the three shown scan types uses the URG flag?
}{
Xmas
}
\QA{
Why are NULL, FIN and Xmas scans generally used?
}{
Firewall Evasion
}
\QA{
Which common OS may respond to a NULL, FIN or Xmas scan with a RST for every port?
}{
Microsoft Windows
}
}


\T{Scan Types: ICMP Network Scanning}{
When connecting for the first time to an unknown machine, it is advisable to obtain an idea of how the network is structured by assessing which IP addresses contain active hosts.\\
Nmap can map the network by performing a "ping sweep", which consists of Nmap sending an ICMP packet to each possible IP address of the network and marking it as "alive" when said IP address responds. Unfortunately, this is not always accurate, but serves as a good base to start moving on from.\\

This ping sweep has the switch \cd{-sn}, and the general syntax
\codenl{nmap -sn <targets>, \textnormal{e.g}}
\codenl{nmap -sn 192.168.0/24}
The \cd{-sn} switch tells Nmap not to scan ports and hence relying on ICMP echo packets or ARP requests on a local network when run with root privileges.\\
It will also send a TCP SYN packet to port 443 (HTTPS) and a TCP ACK resp. TCP SYN if not run as root to port 80 (HTTP) of the target.
\QA{
How would you perform a ping sweep on the 172.16.x.x network (Netmask: 255.255.0.0) using Nmap? (CIDR notation)
}{
The beginning of the command must logically be \cd{nmap -sn}, and to tell Nmap to scan all addresses beginning with \cd{176.16} we tell nmap to keep the first 16 bits as per CIDR-style addressing, i.e
\codenl{nmap -sn 176.16.0.0/16}
}
}


\T{NSE Scripts: Overview}{
This task introduces the Nmap Scripting Engine (NSE), written in the Lua programming language, which is an enhancement for Nmap that can be used to a myriad of things, from vulnerability scans to the automatization of the actual exploit they reveal possible.\\
Best suited for reconnaisance, it offers many categories: \\
\begin{itemize}
\item \cd{safe:} Doesn't affect the target
\item \cd{intrusive:} Does affect the target
\item \cd{vuln:} Scan for vulnerabilities
\item \cd{exploit:} Attempt to exploit a vulerability
\item \cd{auth:} Attempt to bypass authentication for running services, e.g log into an FTP server anonimously
\item \cd{brute:} Attempt to bruteforce credentials for running services
\item \cd{discovery:} Attempt to query running services for further information about the network, e.g query an SNMP server
\end{itemize}

\QA{
What language are NSE scripts written in?
}{
Lua
}

\QA{
Which category of scripts would be a very bad idea to run in a production environment?
}{
\cd{intrusive}
}
}

\T{NSE Scripts: Working with the NSE}{
Coming back to the \cd{--script} switch we used as \cd{--script=vuln} in the 3rd task of this room, one can configure any other cateogry exactly the same way, e.g as \cd{--script=safe}.\\
In a general fashion we run \cd{--script=<script-name>}, even with multiple scripts in parallel when separated by a comma.\\
If any arguments such as credentials for an authenticated vulnerability are needed they can be passed with the switch \cd{--script-args}. They take the form \cd{<script-name>.<argument>}\\
A full example of the script \cd{http-put}, which requires the URL to upload the file to and the file's location on the current disk, would look like this:\\
\codenl{nmap -p 80 --script http-put --script-args http-put.url='/dav/shell.php',http-put.file='./shell.php'}

One can further access the help menu for the corresponding script via \cd{nmap --script-help <script-name>}

\QA{
What optional argument can the \cd{ftp-anon.nse} script take?
}{
Looking at \href{https://nmap.org/nsedoc/}{THIS} list of scripts and arguments we locate the \cd{ftp-anon} script and see \cd{ftp-anon.maxlist} as single possible argument, hence resulting in \cd{\textbf{maxlist}} as answer. 
}
}

\T{NSE Scripts: Searching for Scripts}{
In order to run the scripts from the previous task, we first need to locate them.\\
Under Linux, all scripts are located by default under \cd{/usr/share/nmap/scripts/script.db}, which is not strictly speaking a database, but rather a formatted text, e.g 
\codenl{Entry$\quad$ \{ filename = "smb-os-discovery.nse", categories = \{ "default", "discovery", "safe", \} \} }
We can also \cd{grep} through this database or \cd{ls} on the file with placeholders (*) as free text if needed.\\
In case a script we need is not yet installed, running 
\codenl{sudo apt update \&\& sudo apt install nmap} should update Nmap, or else
\codenl{sudo wget -O /usr/share/nmap/scripts/<script-name>.nse https://svn.nmap.org/nmap/scripts/<script-name>.nse}
followed by 
\cd{nmap --script-updatedb}
will download the desired script and update the database

\QA{
 Search for "smb" scripts in the /usr/share/nmap/scripts/ directory using either of the demonstrated methods.
What is the filename of the script which determines the underlying OS of the SMB server?
}{
After running a simple \cd{grep "smb" /usr/share/nmap/scripts/script.db} and looking through all results gives us a strong hint that the solution is \textbf{\cd{smb-os-discovery.nse}}
}

\QA{
Read through this script. What does it depend on?
}{
Reading through \cd{less /usr/share/nmap/scripts/smb-os-discovery.nse} we find the categories \@ usage, \@ output, \@ xmloutput and right below the author, license, categories and \textbf{\cd{dependencies = \{"smb-brute"\}}}
}
}

\T{Firewall Evasion}{
The default firewall of the Windows OS blocks all incoming ICMP packets, hence making it impossible to ping the target, or rather said ping will result in Nmap marking the host as dead and skipping it.\\
In order to avoid this undesired configuration, we can use the switch \cd{-Pn} to tell Nmap not to ping the targets and considering them all as alive before scanning them.\\
This comes with the downside of a way longer scanning time as Nmap will keep double checking every specified port even in the case of actually dead hosts.\\
One can still note the host activity via ARP requests when scanning from the local network.\\

Further useful switches for firewall evasion:
\begin{itemize}
\item \cd{-f}: Fragments the packets making it less likely to be detected by a firewall.
\item \cd{--mtu <number>}: Alternative to \cd{-f} but with more control over the size of the packets. This number must be a multiple of 8.
\item \cd{--scan-delay <time> ms}: Used to add a delay between packets, useful for unstable networks and for evading time-based firewall/IDS triggers.
\item \cd{--badsum} Used to generate an invalid checksum for packets and assert the pressence of a firewall/IDS if this packet is answered, as any real TCP/IP stack would drop a packet with bad checksum.
\end{itemize}

\QA{
Which simple (and frequently relied upon) protocol is often blocked, requiring the use of the -Pn switch?
}{
ICMP
}

\QA{
[Research] Which Nmap switch allows you to append an arbitrary length of random data to the end of packets?
}{
A quick google search gives us the answer:\\
\cd{\textbf{---data-length}<\# of bytes>}
}
}

\T{Practical}{
Finally, we arrive at a testing task, for which we'll use the machine deployed in Task 1.\\
The question-answer dynamics will provide the guidance for the next steps:\\
\QA{
Does the target (10.10.207.2)respond to ICMP (ping) requests (Y/N)?
}{
Using \cd{man nmap | grep ICMP} we see \cd{-PE; -PP; -PM} as ICMP ping types.\\
We perform an ICMP ping: \codenl{sudo nmap -vv -PE 10.10.207.2}
and see as part of the output:\\
\cd{Note: Host seems down. If it is really up, but blocking our ping probes, try -Pn}, which we'll have to do for any further enumeration on the target.\\
Hence, the target does not respond to ICMP requests and the answer is \textbf{N}
}

\QA{
Perform an Xmas scan on the first 999 ports of the target -- how many ports are shown to be open or filtered?
}{
The corresponding command is 
\codenl{sudo nmap -vv -Pn -sX 10.10.207.2 -p 1-999}
and after some waiting time (the \cd{-Pn} switch treats all ports as alive, resulting in more waiting time), we see 
\codenl{All 999 scanned ports on 10.10.207.2 are open|filtered because of 999 no-responses}
Hence the answer must be \textbf{999}
}

\QA{
There is a reason given for this -- what is it?

Note: The answer will be in your scan results. Think carefully about which switches to use -- and read the hint before asking for help!
}{
Reading the above output we can clearly see that the reason for this classification is \textbf{\cd{no-response}}
}

\QA{
Open Wireshark (see Cryillic's Wireshark Room for instructions) and perform a TCP Connect scan against port 80 on the target, monitoring the results. Make sure you understand what's going on.
}{
This is only a checkbox to be clicked after looking at the results from a Wireshark scan. This is best done within the deployed AttackBox
}

\QA{
Deploy the ftp-anon script against the box. Can Nmap login successfully to the FTP server on port 21? (Y/N)
}{
Running the command
\codenl{sudo nmap -vv -Pn --script=ftp-anon 10.10.207.2}
we see as part of the output, together with further 4 responsive ports (53:domain, 80:http, 135:msrpc, 3389:ms-wbt-server), the result:\\
\cd{21/tcp	open		ftp	syn-ack ttl 127\\
|	ftp-anon: Anonymous FTP login allowed (FTP code 230)\\
|\_ Can't get directory listing: TIMEOUT}\\
so a remote FTP login is possible and the answer is \textbf{Y}
}
}

After terminating the machine and reading the conclusion, we move on to the next topic, \nameref{Network_Services}