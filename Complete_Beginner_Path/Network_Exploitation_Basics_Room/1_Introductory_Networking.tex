\addcontentsline{toc}{subsubsection}{Introductory Networking}
\subsubsection*{Introductory Networking}
\T{Introduction}{
In this room we'll cover the OSI and TCP/IP models, their practical application and some basic networking tools.
}

\T{The OSI Model: An Overview}{
OSI stands for Open Systems Interconnection, and this model is a standard for didactical purposes more than having an actual application. Nevertheless it fulfils this purpose better than "real-world" TCP/IP due to its simplicity.\\
The OSI model is divided into seven layers.\\
A mnemonic to learn the order can be Anxious Pale Shakespeare Treated Nervous Drunks Patiently.\\
The layers are:\\
\begin{mune}
\item \textbf{Application:}\\
		Provides networking options via an interface to transmit data to applications running on a computer.
\item \textbf{Presentation:}\\
		Data received from the previous layer is in a format the application understands but not necessarily in a way the receiving computer would understand.\\
		This layer translates the data to deal with this problem and encrypts, compresses or transform the data in any other needed way
\item \textbf{Session:}\\
		Takes correctly formatted data and tries to establish a connection with the receiving end across the network. If possible, this later mantains it and synchronises communications with the corresponding layer at the receiving end.\\
		Each session is unique to the communication, hence one can make multiple requests to different endpoints at the same time without error.
\item \textbf{Transport:}\\
		This layer chooses the protocol over which the data is to be transmitted, the two most common ones being TCP (Transmission COntrol Protocol) and UDP (User Datagram Protocol).\\
		Their difference relies on the way of the connections, TCP maintaining it for the duration of the whole communication and ensuring the correct transmission of all data at an acceptable communication. It is used when integrity is more important than speed, such as when loading a website or transferring files.\\
		UDP, on the other hand, sends packages at the receiving end without double checking for reception, such as videocalls and streaming services.\\
		After selecting the protocol, the data is fractioned to easen their transfer. 
\item \textbf{Network:}\\
		This layer interprets the destination of the communication request, e.g finding the best route for the given IP address or other logical (IP) addresses. These addresses are still software controlled and provide order to networks, the most common of them being the IPV4 format. 
\item \textbf{Data Link:}\\
		This layer is in charge of the physical addressing of the transmission adding to the data packet containing the IP address coming from the above layer the physical address (MAC) of the receiving end. This Media Access Control (MAC) address is given by the Network Interface Card (NIC) every network enabled computer comes with.\\
		A MAC address can't be changed, but can be spoofed.This address is the one used to identify the receiver of the information.\\
		Further, this layer presents the data in a transmission-suitable format and is in charge of checking the integrity of the data when on the receiving end.
\item \textbf{Physical:}\\
		This layer is actually down to the hardware of the computer. It converts the binary data into electrical signal and transmit them across the network and the other way around when receiving data. 
\end{mune}
The questions are mostly about the assignment of a task to the corresponding layer of the OSI model:\\
\QA{
Which layer would choose to send data over TCP or UDP?}{
4}
\QA{
Which layer checks received information to make sure that it hasn't been corrupted?
}{
2}
\QA{
In which layer would data be formatted in preparation for transmission?
}{
2}
\QA{
Which layer transmits and receives data?
}{
1}
\QA{
Which layer encrypts, compresses, or otherwise transforms the initial data to give it a standardised format?}{
6}
\QA{
Which layer tracks communications between the host and receiving computers?
}{
5}
\QA{
Which layer accepts communication requests from applications?
}{
7}
\QA{
Which layer handles logical addressing?}{
3}
\QA{
When sending data over TCP, what would you call the "bite-sized" pieces of data?
}{
Segments}
\QA{
[Research] Which layer would the FTP protocol communicate with?}{
Since it communicates with the FTP client, a special application designed for that purpose, it stays on layer 7}
\QA{
Which transport layer protocol would be best suited to transmit a live video?}{
Since we'd rather care about the "live" part of the video, time synchronisation is preferred over quality of data, hence UDP is the better protocol.}
}

\T{Encapsulation}{
As the data are passed down each layer, more information related to the corresponding layer is added on to the start of the transmission, such as source and destination IP addresses, protocol being used, etc.\\
The Data Link layer adds information at the end to verify ensure integrity of the transmission.\\
This process is called encapsulation.\\
Within said encapsulation, the data changes its name when going down the model:\\
At Layer 4 it is called Segments (TCP) or Datagrams (UDP); at Layer 3, Packets; at Layer 2, Frames and at Layer 1, Bits.\\
The receiving computer takes off the layer-related header until it arrives at the desired information in a process called de-encapsulation and hence providing a standardised communication system between network-enabled computers.\\
The questions on this task are only a reading comprehension of the above contents.
\QA{How would you refer to data at layer 2 of the encapsulation process (with the OSI model)?}
{Frames}

\QA{How would you refer to data at layer 4 of the encapsulation process (with the OSI model), if the UDP protocol has been selected?}{
Datagrams}
\QA{
What process would a computer perform on a received message?}{
De-encapsulation}
\QA{
Which is the only layer of the OSI model to add a trailer during encapsulation?}{
Data Link}

\QA{
Does encapsulation provide an extra layer of security (Aye/Nay)?}{
Aye}
}

\T{The TCP/IP Model}{
This is a model coming from real-world networking, even though it is some years older. This model is more condensed than the OSI model and hence worse for learning purposes.\\
In some recent sources the TCP/IP model is split into 5 models, dividing the last layer (Network Interface) into the same two layers as the OSI model: Data Link and Physical.\\ 
In most sources it still only consists of four layers:\\
\begin{mune}
\item \textbf{Application:}\\
It corresponds to the 7th, 6th and 5th layers (Application, Presentation and Session) of the OSI model.
\item \textbf{Transport:}\\
Same as the Transport layer of the OSI model.
\item \textbf{Internet:}\\
Corresponds to the Network layer of the OSI model.
\item \textbf{Network Interface:}\\
Corresponds to the Data Link and Physical layers of the OSI model, but, as stated above, some inofficial models divide this last layer in the two coming from the OSI model.
\end{mune}
The encapsulation and de-encapsulation work in the same way as in the OSI model.\\
What the layers actually mean are a suite of protocols to carry out the exchange, the two most important of them giving name to the model: \\
\textbf{TCP} stands for Transmission Control Protocol, which controls the flow of data between two endpoints, \textbf{IP} for Internet Protocol, which controls how packets are addressed and sent.\\
TCP is a connection-based protocol which as such requires a stable connection performed via a three-way handshake.\\
First, the connecting computer sends a request signalising the desire to establish a connection, i.r to synchronise.\\
Hence this request contains a SYN bit, which serves as first contact.\\
Then, the server answers with an package containing the SYN bit and another aknowledgment bit ACK.\\
Last, the connecting computer answers with the ACK bit confirming the connection.\\
This way, data can be reliably transmitted between the computers with re-sending of any lost data.\\
\QA{Which model was introduced first, OSI or TCP/IP?
}{
TCP/IP was introduced in 1982, and even though the OSI model is newer and more helpful for visualization, the TCP/IP model is still broader in use.}
\QA{
Which layer of the TCP/IP model covers the functionality of the Transport layer of the OSI model (Full Name)?
}{
Transport
}
\QA{
Which layer of the TCP/IP model covers the functionality of the Session layer of the OSI model (Full Name)?
}{
Application
}
\QA{
The Network Interface layer of the TCP/IP model covers the functionality of two layers in the OSI model. These layers are Data Link, and?.. (Full Name)?
}{
Physical
}
\QA{
Which layer of the TCP/IP model handles the functionality of the OSI network layer?
}{
Internet
}
\QA{
What kind of protocol is TCP?
}{
Connection-based
}
\QA{
What is SYN short for?
}{
synchronise
}
\QA{
What is the second step of the three way handshake?
}{
SYN/ACK
}
\QA{
What is the short name for the "Acknowledgement" segment in the three-way handshake?
}{
ACK
}
}
\T{Networking Tools: Ping}{
In this task, we are gonna be looking at the \cd{ping} command, used to test if a connection to a remote source, usually a website but also a computer in the home network, is possible.\\
Pings work using the ICMP protocol, a less-known variant of the TCP/IP protocol which works on the Network later of the OSI model and the Internet layer of the TCP/IP model.\\
Its syntax is \codenl{ping < target >}.\\
The questions of this Task are all related to this command and can be solved by looking at the \cd{man ping} manual or via \cd {ping -h}.
\QA{
What command would you use to ping the bbc.co.uk website?
}{
\cd{ping bbc.co.uk}
}
\QA{
Ping muirlandoracle.co.uk
What is the IPv4 address?
}{
217.160.0.152
}
\QA{
What switch lets you change the interval of sent ping requests?
}{
\cd{ -i}
}
\QA{
What switch would allow you to restrict requests to IPv4?
}{
\cd{-4}
}
\QA{
What switch would give you a more verbose output?
}{
\cd{-v}
}
}
\T{Networking Tools: Traceroute}{
Following the ping request we can use \cd{traceroute}, which outputs the path our request takes until reaching the target machine, listing the servers and connections the request goes through.\\
The syntax for the command is \cd{traceroute <destination>}\\
For Windows, \cd{tracert} operates with the same ICMP protocol as \cd{ping} and the Unix pendant over UDP, though changeable via switches on the command.\\
The first question is a checkbox to click after tracing the route to tryhackme.com. The rest can be answered after taking a look at \cd{man traceroute}.
\QA{
What switch would you use to specify an interface when using Traceroute?
}{
\cd{-i}
}
\QA{
What switch would you use if you wanted to use TCP SYN requests when tracing the route?
}{
\cd{-T}
}
\QA{
[Lateral Thinking] Which layer of the TCP/IP model will traceroute run on by default (Windows)?
}{
As the \cd{traceroute} command follows a network request, it must run on the \textbf{Internet} layer.
}
}
\T{Networking Tools: WHOIS}{
In order not to have to remember every IP address of the sites we want to visit, we can use domains leased out by domain registrars companies.\\
In order to know whose name a domain is registered to, we use the command \cd{whois} in the syntax \cd{whois <domain>}.\\
The questions are all related to \cd{whois} queries on different domains: First on facebook.com, then on microsoft.com, each then having a checkbox as an answer
\QA{
What is the registrant postal code for facebook.com?
}{
94025
}
\QA{
When was the facebook.com domain first registered (Format: DD/MM/YYYY)?
}{
29/03/1997
}
Then on \cd{microsoft.com}:
\QA{
Which city is the registrant based in?
}{
Redmond
}
\QA{
[OSINT] What is the name of the golf course that is near the registrant address for microsoft.com?
}{
Looking at One Microsoft Way, Redmond, WA, 98052, we see the "Bellevue Golf Course" northwest of the address,hence the answer is \textbf{Bellevue}.
}
\QA{
What is the registered Tech Email for microsoft.com?
}{
Looking at the output of said \cd{whois microsoft.com} we see \textbf{\cd{msnhst@micorsoft.com}} as desired answer.
}
}
\T{Networking Tools: Dig}{
In order to convert the URL to an IP address we use a TCP/IP protocol called DNS (Domain Name System).\\
At a basic level, we request a special server to translate the website we want to access into a viable IP address we then send a request to.\\
This works by first checking the local "hosts file" to see if an explicit IP $\rightarrow$ Domain mapping has been created.\\
Though being older and less commonly used than DNS it takes precedence in the search order. When no mapping is found, the computer checks in its local DNS cache if a preexisting mapping already was saved. \\
Else, the connecting computer will send a request to a recursive DNS server automatically known to the router on the network. These are property of Internet Service Providers (ISPs) but some companies such as OpenDNS also control recursive servers, such that every computer automatically knows where to send the request for information.\\

These recursive DNS servers store a cache for popoular domains, but to access different websites they pass the request on to a root name server. Before 2004 there were only 13 such servers, and all further such servers added can still be accessed using those same 13 addresses. They keep track of the DNS servers in the next level down and choose the best route to redirect the passed request. \\

These lower level servers are called Top-Level Domain (TLD) servers. They are split up into extensions given by the ending of the requested URL, such as \cd{.com, .es, .eu}, etc.\\
TLD servers further keep track of the next level down: Authoritative name servers, and passes the requests down to them.\\

Authoritative name servers are used to store DNS records for domains directly sucht that every domain will have its DNS in an Authoritative name server.\\
Once the request reaches the ANS, it will send the info back to the requesting computer to allow for a connection to the necessary IP address.\\

This is where the command \cd{dig} comes into place. It queries recursive DNS servers of our choice for information about domains in the syntax:
\codenl{dig <domain> @<dns-server-ip>}
This command is very useful for network troubleshooting.\\
For this room, we will need the \cd{ANSWER} section, which tells us about the success of the transmission of a full answer containing the IP address for the domain name we queried.\\
The \cd{dig} command further tells us the TTL (Time To Live) of the queried DNS record in the local cache of the computer, to be measured in seconds and read in the second column of the \cd{dig} command.\\

As usual in such theory-heavy tasks, the questions are mostly a reading comprehension:
\QA{
What is DNS short for?
}{
Domain Name System
}
\QA{
What is the first type of DNS server your computer would query when you search for a domain?
}{
Recursive
}
\QA{
What type of DNS server contains records specific to domain extensions (i.e. .com, .co.uk*, etc)*? Use the long version of the name.
}{
Top-Level Domain
}
\QA{
Where is the very first place your computer would look to find the IP address of a domain?
}{
Hosts File
}
\QA{
[Research] Google runs two public DNS servers. One of them can be queried with the IP 8.8.8.8, what is the IP address of the other one?
}{
Running a simple search we find the other one under \textbf{8.8.4.4}
}
\QA{
If a DNS query has a TTL of 24 hours, what number would the dig query show?
}{
The TTL being measured in seconds, it would show $24*60*60 =$ \textbf{86400}
}
}
This task ends the room with some further reading recommendations and a redirection to the TryHackMe discord, as well as an amazon link to the "CISCO Self Study Guide" by Steve McQuerry.