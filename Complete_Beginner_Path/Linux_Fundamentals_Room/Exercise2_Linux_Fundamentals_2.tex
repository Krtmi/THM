\addcontentsline{toc}{subsubsection}{Exercise 2: Linux Fundamentals 2}
\subsubsection*{Exercise 2: Linux Fundamentals 2}
\label{Exercise 2: Linux Fundamentals 2}
\externaldocument{Complete_Beginner_Path/Room2_Linux_Fundamentals/Exercise3_Linux_Fundamentals_3}
In this section we'll connect to machines remotely, perform more useful commands to get the access permissions and much more and run the first scripts.\\
The first task requires no further actions than clicking the checkbox
\begin{task}[Accessing Your Linux Machine Using SSH (Deploy)]
To access the THM machine via ssh we only need the command\\
\code{ssh tryhackme$@$[IP\_ ADDRESS\_ OF\_ THE\_ MACHINE]}\\
This can be done from the AttackBox inside the Exercise or via \code{openvpn} after \codenl{sudo openvpn username.ovpn}
\end{task}
\begin{task}[Introduction to flags and switches]
In this task we review the \code{ls} command, especially its switches \code{-a, --all, --help}.\\
In order to learn more about it we also use (introduce?) the \code{man} (manual) command, which provides a small explanation of the previously passed command

THe first question is only a checkbox, and the task comprises the following questions: 
\begin{Q}
What directional arrow key would we use to navigate down the manual page?
\end{Q}
\begin{A}
down
\end{A}

\begin{Q}
What flag would we use to display the output in a "human-readable" way?
\end{Q}
\begin{A}
\code{-h}, i.e \code{ls -h} outputs the contents of the current folder in a human readable way
\end{A}

\end{task}

\begin{task}[Filesystem Interaction Continued]
In this task, more basic commands for the Filesystem are introduced:\\ 
\begin{tabular}{r|c|l}
Command & Full Name & Purpose\\
\hline
\cd{touch}& touch & Create file\\
\cd{mkdir} & make directory & Create a folder\\
\cd{cp} & copy & Copy a file or folder\\
\cd{mv} & move & Move a file or folder\\
\cd{rm} & remove & Remove a file or folder\\
\cd{file} & file & Determine the type of a file\\
\end{tabular}

To create a file, it suffices to use the command \codenl{touch testfile}
Similarly to create and remove a folder or other file, or to get the type of a file via \cd{file testfile}.\\
In order to move or copy files, one needs to pass a second argument to the command as a destination.
\begin{Q}
How would you create the file named "newnote"?
\end{Q}
\begin{A}
\codenl{touch newnote}
\end{A}
\begin{Q}
On the deployable machine, what is the file type of "unknown1" in "tryhackme's" home directory?
\end{Q}
\begin{A}
Using \cd{file unknown1} we get thet its type is ASCII text
\end{A}
\begin{Q}
How would we move the file "myfile" to the directory "myfolder"?
\end{Q}
\begin{A}
We'd use the command \codenl{mv myfile myfolder}
\end{A}
\begin{Q}
What are the contents of this file?
\end{Q}
\begin{A}
Using \codenl{cat myfile} we get the flag 
\flag{THM\{FILESYSTEM\}}
\end{A}
The last question only encourages the user to keep practicing and is resolved by clicking on a checkbox.
\end{task}

\begin{task}[Permissions 101]
In this task we explore the permissions over files and user differences regarding them:\\
With the command \cd{ls -lh} we can see the owner of each file and the permissions associated to them, the letters \cd{r, -w, --x} standing for Read, Write, Execute.\\
Using the command \codenl{su user2} we can switch users to \cd{user2} after we deliver the corresponding password. Further specifying the switch \cd{-l, --login} we start a shell within the user's directory.\\
The questions of this task deal with the changing of users:
\begin{Q}
On the deployable machine, who is the owner of "important"?
\end{Q}
\begin{A}
Using \codenl{ls -lah} we see that the owner on important is \cd{user2}
\end{A}
\begin{Q}
What would the command be to switch to the user "user2"?
\end{Q}
\begin{A}
\cd{su user2} resp. \cd{su user2 -l}
\end{A}
The 3rd question is only a checkbox to click after switching users\\
\begin{Q}
Output the contents of "important", what is the flag?
\end{Q}
\begin{A}
Using \cd{cat important} we get the flag: 
\flag{THM\{SU\_ USER2\}}
\end{A}
\end{task}

\begin{task}[Common Directories]

Here some of the common directories to all Linux systems are displayed: 
\begin{itemize}
\item \cd{/etc}:\\
Short for etcetera, commonplace location to system files for the OS.\\
Especially important are the files \cd{passwd, shadow}, which show how the system stores the passwords for each user in SHA512.
\item \cd{/var}:\\
Short for variable data, stores data that are frequently accessed or written, e.g. log files under \cd{/var/log} or non user-associated data
\item \cd{/root}:\\
Home directory for the \cd{root} system user, not to be confused with \cd{/home} or \cd{/home/root}. 
\item \cd{/tmp}: \\
Short for temporary, volatile and used to store data used only once or twice. Deleted after restart.\\
\textbf{NOTE:} Used in pentesting to store enumeration scripts and similar tools since any user can write to this folder by default.
\end{itemize}

The first task consists only of a checkbox.
\begin{Q}
What is the directory path that would we expect logs to be stored in?
\end{Q}
\begin{A}
\cd{/var/log}
\end{A}
\begin{Q}
What root directory is similar to how RAM on a computer works?
\end{Q}
\begin{A}
\cd{/tmp}
\end{A}
\QA	{Name the home directory of the root user}
	{\cd{/root}}

The final task is also only a checkbox with the instruction to navigate through these folders in the deployed Linux machine. 
\end{task}
The last two Tasks consist of a recapitulation of the contents of this Room (\cd{ssh}, flags and switches on commands, users 101 and important Linux directories) and a diversion to the next room \nameref{LinuxFund3}.