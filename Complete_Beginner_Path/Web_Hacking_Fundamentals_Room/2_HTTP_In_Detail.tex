\addcontentsline{toc}{subsubsection}{HTTP in detail}
\subsubsection*{HTTP in detail}
\T{What is HTTP(S)?}{
HTTP stands for HyperText Transfer Protocol and is the set of rules used to communicate with web servers to transmit data, be it in the form of HTML, multimedia, etc.\\
HTTPS is the secure version of HTTP, as it is encrypted both covering the authenticity and confidentiality parts of security.\\
\QA{
What does HTTP stand for?
}{
HyperText Transfer Protocol
}
\QA{
What does the S in HTTPS stand for?
}
{
Secure
}
\QA{
On the mock webpage on the right there is an issue, once you've found it, click on it. What is the challenge flag?
}{
We click on the striked lock on the website on the right denoting the absence of https or, in general, encryption, and get a pop up window with the flag:
\F{THM\{INVALID\_HTTP\_CERT\}}
}
}
\T{Requests And Responses}{
When accessing a website, the browser makes requests to a webserver to load the site contents and get eventual responses. \\
In order to allow the browser to communicate with the server, we provide it with a target URL (Uniform Resource Locator), roughly speaking an instruction on how to access our target.\\
The general syntax is as follows:
\codenl{<scheme>://[user]$@$<host/domain>:[port]/[path][?query string][\#fragment]}
which in a real world example would look like this:
\codenl{http://user:password$@$tryhackme.com:80/view-room?id=1\#task3}
Breaking down the syntax:
\begin{itemize}
\item Scheme: The scheme gives the browser instructions on which protocol to use to access the desired data, e.g HTTP(S), FTP...
\item User: The user provides the authentication credentials to log in, which can directly be provided in the URL.
\item Host: The host gives the domain name or IP address of the server we want to contact.
\item Port: We can select any port we want, though default is 80 for http and 443 for https.
\item Path: The file name or location of the desired ressource.
\item Query string: Extra information additional to the path.\\We can use e.g \cd{/blog?id=1} to tell the blog path to access the ressource with id of 1.
\item Fragment: This references part of the target content, especially in websites with long contents, when we want to access only part of it instead of the head of the page.
\end{itemize}
A request is composed of the request method, the page being requested and the HTTP protocol version.\\
The minimal HTTP request is \cd{GET / HTTP/1.1}, which doesn't result in much information being loaded. In order to load further contents we will send more request data in the form of headers.\\
An example request has the form:\\
\cd{
GET / HTTP/1.1\\
Host: tryhackme.com\\
User-Agent: Mozilla/5.0 Firefox/87.0\\
Referer: https://tryhackme.com\\
}
The request breaks up linewise as follows:
\begin{enumerate}
\item Sends the GET method, requesting the home page by sending a ``\cd{/}'' as site request and specifies the protocol and its version as \cd{HTTP 1.1}
\item Calls the requested website as host.
\item Tells the browser we are using and its version, in this case Mozilla Firefox version 87.0
\item The page which derived us here is \cd{https://tryhackme.com}
\item A blank line to communicate the end of the request.
\end{enumerate}
The response then is made up from the website we want to load, if our request was successful.\\
A basic example of the response could be as follows:\\
\cd{
HTTP/1.1 200 OK\\
Server: nginx/1.15.8\\
Date: Fri, 09 Apr 2021 13:34:03 GMT\\
Content-Type: text/html\\
Content-Length: 98\\
\newline
<html>\\
<head>\\
\indent    <title>TryHackMe</title>\\
</head>\\
<body>\\
\indent    Welcome To TryHackMe.com\\
</body>\\
</html>\\
}
Broken down linewise again:
\begin{enumerate}
\item Protocol version and HTTP Status Code: Here 200 OK means a successful request. 
\item Web server software and version number.
\item Date, time and timezone of the server.
\item Content-Type: what sort of information is going to be sent.
\item Content-Length: Serves to confirm no data is missing.
\item Blank line after the HTTP response
\item[7.-14.] Requested webpage information 
\end{enumerate}

\QA{
What HTTP protocol is being used in the above example?
}{
HTTP/1.1
}
\QA{
What response header tells the browser how much data to expect? 
}{
Content-Length
}
}
\T{HTTP Methods}{
HTTP methods determine the kind of interaction we as client want to perform on or with the server.\\
The most basic HTTP methods are the following:
\begin{itemize}
\item GET: Used to get information from the webserver.
\item POST: Used to submit data to the webserver, including the creation of new records.
\item PUT: Used to submit data to update information on the webserver
\item DELETE: Used to delete data from the webserver.
\end{itemize}
\QA{
What method would be used to create a new user account?
}{
\cd{POST}
}
\QA{
What method would be used to update your email address? 
}{
\cd{PUT}
}
\QA{
What methoud would be used to remove a picture you've uploaded to your account? 
}{
\cd{DELETE}
}
\QA{
What method would be used to view a news article? 
}{
\cd{GET}
}
}
\T{HTTP Status Codes}{
As seen in the task before, the first line of an HTTP response is the HTTP status code. Depending on their initial number, they can be assigned to different cases: 
\begin{itemize}
\item 100-199: The first part of the request has been accepted, the client shall continue sending the rest of their request. Rather in disuse.
\item 200-299: Successful request.
\item 300-399 - Redirection: Used to redirect the request to another resource, be it a different webpage or website.
\item 400-499 - Client Errors: Error with the client request.
\item 500-599 - Server Errors: Errors on the server side of the request handling.
\end{itemize}
Some of the most common status codes accross all categories are the following:
\begin{itemize}
\item 200-OK: Request completed successfully.
\item 201-Created: Resource (e.g user, blog entry) created successfully.
\item 301-Moved permanently: The originally requested website has moved permanently, indicates where to look for the new address.
\item 302-Found: The originally requested website has moved temporarily, redirects to the current new direction of the website. 
\item 400-Bad Request: The request sent by the client is either incomplete or wrong.
\item 401-Not Atuhorised: Until login with the right credentials this request is blocked. 
\item 403-Forbidden: This resource is blocked, no matter the credentials.
\item 404-Page Not Found: The requested page does not exist.
\item 405-Method Not Allowed: Wrong combination of HTTP method and intention, e.g using a \cd{GET} request to create an account.
\item 500-Internal Service Error: The server has some kind of error with the sent request it doesn't know how to handle.
\item 503-Service Unavailable: The server cannot handle the request being down for maintenance or overloaded.
\end{itemize}
\QA{
What response code might you receive if you've created a new user or blog post article?
}{
201
}
\QA{
What response code might you receive if you've tried to access a page that doesn't exist?
}{
404
}
\QA{
What response code might you receive if the web server cannot access its database and the application crashes?
}{
503
}
\QA{
What response code might you receive if you try to edit your profile without logging in first?
}{
501
}
}
\T{Headers}{
As we saw before, headers are the aditional information sent to the server to specify some information about the data being transmitted. They are not necessary, but very rarely a request without them will result in the expected browsing experience.\\

The most common Request headers are: 
\begin{itemize}
\item Host: Specifies the requested website in case the web server hosts multiple. If not passed, the request will be handled as passing the default website for the server
\item User-Agent: Tells the browser software and version number to help format the requested website and handle HTML, JS and CSS elements.
\item Content-Length: Ensures no data loss has happened along the way
\item Accept-Encoding: Tells the supported compression methods to help with data transmission. 
\item Cookie: Help the server remember our information.
\end{itemize}

The most common Response headers are:
\begin{itemize}
\item Set-Cookie: Sets the cookie information for later requests.
\item Cache-Control: Tells how long to store the response in the browser's cache. 
\item Content-Type: Tells the client which type of data are being returned, e.g HTML, CSS, JS, Images, PDF, etc. such that the browser knows how to process the data. 
\item Content-Encoding: Specifies the compression method sfor transmission.
\end{itemize}
\QA{
What header tells the web server what browser is being used? 
}{
User-Agent
}
\QA{
What header tells the browser what type of data is being returned? 
}{
Content-Type
}
\QA{
What header tells aewsffffeejeawweasfsewwesafefwsewseefwsewsfeej,wefswesaewsafwaeswesawesjeewaeswesafseafwasweaaffffffffffffffffffeeeawsewseafffffffffaaaaaaaaafasewaafffffffffffwesafwswwwswewseaeaaaaaaaaaaaaaaaaaaaaaaaaaaaaaaaaaaaaaaaaafffffffffffffffffffffffffffffffffffffffffffffffffffffffffffffffffffffffffffffffffffffffffffffffffffffffffffffffffffffffffffffffffffffffffffffffffffffffffffffffffffffffffffffffffffffffffffffffffffffffffffffffffffffffffffffffffffffffffffffffffffffffffffffffffffffffffffffffffffffffffffffffffffffffffffffffffffffffffffffffffffffffffffffffffffffffffffffffffffffffffaffffffffffffffffaaaaaaaaaaaaaasedfewafssaefewafssefewafssefwewfasjkjjeswewfsdfewsfefwefawsewfsefwewsfaewfasewwfjfefdwsjjefjjjefjjefwsefwsjfejjjjejjjjjfeedfwjejjeewsfefwedfwsjeefwjsjeefjewedfjjjjjjejejjedfwsjedfwjjfefjjedfwjewjefwefwsaeeefewasaeeefefjefdwfaeewfsdfdeefwdefewfsdeefwewefwjfeefwefefwaeesaeewfeeeedeejjeedjwaeeedeedfjefjjedjeedfwjefjeejefaeeedfeafeeeefdwefefwfjeefjefwjjeefjjewsfeefwjeweefweffdeeafweewfwaefwefdweeefjjefwjeedfjeeefjefefeedfjfdeedfjefwefjeffeefdwefjjedfjfjjfjefjjeefwjefdewjeffewjeefsefefjsjjefdwsefwsefweweafefweweewseaeewfedwefdwsfaefwsewsfdjjedjewffdeeffaewfjjefefdwsdefwsefdwsefwsefwsfewjjefjjjedfwjjjjedfwjedeedfweffewsjedfjefewjjefjjefejefdewfejejeffeeefweeefejjfefwsjefdwfewewsfdefdwsewsffedwsewsfaaeeeswfdefdwsewsafefwsefdwsefdwseseefswwwwwwwwwwwwwwwwwwwwwwwwwwwwwewsfaaaaaaaaaaaaaaaaaaaaaaaaaaaaaaaaaaaaaaaaaaaaaaaaaaaaaaaaaaaaaaaaaaaaaaaaaaaaaaaaaaaaaaaaaaaaaaaaaaaaaaaaaaaaaaaaaaaaaaaaaaaaaafejfjeffewedfwfeawsjjejjfthe web server which website is being requested? 
}{
Host
}
}
\T{Cookies}{
Cookies are small pieces of data that are stored on the client computer and saved at server response under the header \cd{Set-Cookie}.\\
Every further request the client makes will send the cookie back to the server with the \cd{Cookie} header to keep track locally of the previous requests done, as HTTP being stateless cannot store that information.\\
Cookies serve, among other things, to save website preferences, credentials and much more. They especially serve to authenticate users, as they are usually sent in the form of tokens.\\

To view one's cookies, one only has to open the Web Developer Tools and click on the Network tab. This will show a list of all requested resources, and under the Cookies tab of each request one can find the eventually requested cookies.
\QA{
Which header is used to save cookies to your computer? 
}{
Set-Cookie
}
}
\T{Making Requests}{
This last task is an emulator for HTTP requests and the questions are the arising flags after the correct execution of the requests.
\QA{
Make a GET request to \cd{/room}
}{
In the mock up search bar we complete the search path to \cd{https://tryhackme.com/room} and click ``Go'', seeing a flag as response.\\
For the sake of completeness, our full request was:\\
\cd{
GET /room HTTP/1.1\\
Host: tryhackme.com\\
User-Agent: Mozilla/5.0 Firefox/87.0\\
}
The arising flag:
\F{THM\{\_YOU'RE\_IN\_THE\_ROOM\}}
}
\QA{
Make a GET request to /blog and using the gear icon set the id parameter to 1 in the URL field
}{
\cd{
GET /blog?id=1 HTTP/1.1\\
Host: tryhackme.com\\
User-Agent: Mozilla/5.0 Firefox/87.0\\
}
and we get the flag
\F{THM\{YOU\_FOUND\_THE\_BLOG\}}
}
\QA{
Make a DELETE request to /user/1
}{
\cd{
DELETE /user/1 HTTP/1.1\\
Host: tryhackme.com\\
User-Agent: Mozilla/5.0 Firefox/87.0\\
Content-Length: 0\\
}
and find the flag
\F{THM\{USER\_IS\_DELETED\}}
}
\QA{
Make a PUT request to /user/2 with the username parameter set to admin
}{
\cd{
PUT /user/2 HTTP/1.1\\
Host: tryhackme.com\\
User-Agent: Mozilla/5.0 Firefox/87.0\\
Content-Length: 14\\
\newline
username=admin\\
}
\F{THM\{USER\_HAS\_UPDATED\}}
}
\QA{
POST the username of thm and a password of letmein to /login
}{
\cd{
POST /login HTTP/1.1\\
Host: tryhackme.com\\
User-Agent: Mozilla/5.0 Firefox/87.0\\
Content-Length: 33\\
\newline
username=thm\&password=letmein\\
}
\F{THM\{HTTP\_REQUEST\_MASTER\}}
}
}
This ends the room and we continue with the next one: BurpSuite Basics.