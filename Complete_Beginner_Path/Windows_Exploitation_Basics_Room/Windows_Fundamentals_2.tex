\addcontentsline{toc}{subsubsection}{Windows Fundamentals 2}
\subsubsection*{Windows Fundamentals 2}
\T{System Configuration}{
For this task we load the System Configuration Utility by looking in the search bar for \cd{MSConfig}. \\
This app is used for advanced troubleshooting and to assess in the diagnose of startup issues, as it lists and can launch all procedures defined at startup, as well as redefine the start up modus. \\
Upon launch, we see five tabs: 
\begin{enumerate}
\item General
\item Boot
\item Services
\item Startup 
\item Tools
\end{enumerate}
Under \cd{General}, we can choose among Normal, Diagnostic or Selective for the boot configuration, hence allowing us to debug any potential issues. \\
Under \cd{Boot} we can define different boot options for the OS.\\
Under Services we can see all system services, i.e background-running applications, independent of their activity state. \\
Under \cd{Startup} we are advised to use the Task Manager to enable or disable startup items, as the System Configuration is not a startup management program. \\
Under \cd{Tools} we see the different utilities at our disposal to configure the OS, along with an attached description and a full enumeration of the corresponding command to launch every tool either on the Run or command prompts or directly under the ``Launch'' button.
\QA{
What is the name of the service that lists Systems Internals as the manufacturer?
}{
After listing all services by Manufacturer Name, we find \bf{PsShutdown} to be the answer
}
\QA{
Whom is the Windows license registered to?
}{
We launch the ``About Windows'' Tool and see the license is registered to \bf{Windows User}
}
\QA{
What is the command for Windows Troubleshooting?
}{
We can plainly read it under the Tools tab, namely \\
\cd{C:$\backslash$Windows$\backslash$System32$\backslash$control.exe /name Microsoft.Troubleshooting}
}
\QA{
What command will open the Control Panel? (The answer is  the name of .exe, not the full path)
}{
We see under the Task Manager (or know it from previous times) that the command must be \cd{control.exe}
}
}
\T{Change UAC Settings}{
As covered in the previous Room, the User Account Control can set the permissions for the access of multiple users to potentially harmful programs.\\
Moving the slider in its settings we can check how the OS will react to said potential changes depending on the strictness of our decision.
\QA{
What is the command to open User Account Control Settings? (The answer is the name of the .exe file, not the full path)
}{
We see in the previous Tools tab the path to the UAC Settings plainly, providing the answer: \cd{UserAccountControlSettings.exe}
}
}
\T{Computer Management}{
In the System COnfiguration panel we further see the Computer Management utility, accessible with the \cd{compmgmt} abbreviation or the \cd{compmgmt.msc} command.\\
It has three sections: System Tools, Storage and Sevices and Applications, each of which further subdivides into their corresponding tools and utilities.\\
Under \textbf{System Tools} we see the following:
\begin{itemize}
\item Task Scheduler: with this tool, we can manage the execution time or event catalyzer for different tasks, such as running scripts, applications or further execution of events. These can be scheduled at a certain time, at time intervals, or at events, e.g log in or log off. 
\item Event Viewer: we can use this task to review past events, hence allowing us to audit the activity of the system and diagnose eventual problems. This is where the logs are kept for viewing.\\
It is divided in three parts: on the left we see the event log providers, e.g Windows, Applications and Services, etc.; on the middle we have an overview and summary of events of the chosen provider; and on the right we see the available actions to the corresponding logs.\\
The possibly logged events must be one of: \cd{Error, Warning, Information, Success Audit, Failure Audit}.\\
Of these five ,mostly the last two need further clarification: A \cd{Success Audit} log corresponds to an audited successful security access attempt, e.g a successful logon on the system. \\
A \cd{Failure Audit} log corresponds to a failed event as above, e.g the failure to access a network drive.\\
Windows Logs stores the standard logs, a table of which can be found under \href{docs.microsoft.com}{this link}.\\
They are subdivided into Application logs, its recording determined by the app developer; Security logs, such as logon attempts and events related to resources (e.g files and drives) on the system; System logs, given by system components such as drives' loading logs; and CustomLogs, an umbrella term for any event logged by applications.
\item Shared Folders: here we see the full enumeration of shares and folders shared on the system. The default shares are C\$ and ADMIN\$. One can always view the properties of the share by right-clicking on them.
\item Sessions: here all users connected to the shares can be seen in a list together with their accessed folders and files under Open Files. \\
\item Local Users and Groups: a section already known from the Windows Fundamentals 1 Module
\item Performance: the Performance Monitor, called by the command \cd{perfmon}, is a tool used to view the performance of the system live or at a time from a log. It is mostly used in troubleshooting contexts. 
\item Device Manager: this section lists all hardware on the system and allows us to perform eventual changes such as enabling and disabling peripherics.
\end{itemize}

Under \textbf{Storage} we find the categories Windows Server Backup and Disk Management. The former is slightly self-explanatory and the latter is used for storage tasks, e.g setting up a new drive or adjusting the size of partitions.

Under \textbf{Services and Applications} we can perform advanced tasks on the system services, i.e applications running in the background. Among the possible tasks is WMI (Windows Management Instrumentation) Control, which allows scripting languages to manage Microsoft Windows computers and servers locally and remotely. This is deprecated as of Win10, v.21H1 and substituted by Windows Power Shell

\QA{
What is the command to open Computer Management? (The answer is the name of the .msc file, not the full path)
}{
As we saw above, the command is \cd{compmgmt.msc}
}
\QA{
At what time every day is the GoogleUpdateTaskMachineUA task configured to run? 
}{
Opening the Task Scheduler we see the GoogleUpdateTaskMachineUA with a description stating it to run every day at \textbf{6:15 AM}. The actual running time depends on the time of the machine, as it will most likely be dependent on one of the US time zones.
}
\QA{
What is the name of the hidden folder that is shared?
}{
Under ``Shared Folders'' we find a suspicious folder named \cd{sh4r3dF0Ld3r}
}
}
\T{System Information}{
In this task, we will deal with the System Information tool of the System Configuration panel, callable via \cd{msinfo32.exe}.\\
It provides information about the computer and its components, both hardware and software. It is mostly used for diagnostic purposes.\\
First, it displays a System Summary with general information on the computer's specifications
Its further information is divided into the following categories: \\
\begin{itemize}
\item Hardware Resources: Deemed more advanced than needed at this point, especially for the standard user, we are referred to further reading under \href{https://docs.microsoft.com/en-us/windows-hardware/drivers/kernel/hardware-resources\#:~:text=Hardware\%20resources\%20are\%20the\%20assignable,of\%20bus\%2Drelative\%20memory\%20addresses.}{this} official link.
\item Components: under this category one can read specific information about the peripherics and other devices installed on the computer. 
\item Software Environment: here we can find the information about the software on the computer, both installed and preset. \\
Here we can further see the Environment Variables and Network Connections, although these are noth the only paths leading o them (especially Environment Variables has multiple ways to get to). 
\end{itemize}
\QA{
What is the command to open System Information? (The answer is the name of the .exe file, not the full path)
}{
As we read above, we can open the \cd{msinfo32} util by running \cd{msinfo32.exe}
}
\QA{
What is listed under System Name?
}{
In the ``System Summary'' section, we can plainly read the name \textbf{THM-WINFUN2}
}
\QA{
Under Environment Variables, what is the value for ComSpec? 
}{
Reading as instructed from the Environment Variables tab and resorting to the search bar if needed, we plainly read the value \cd{\%SystemRoot\%$\bsl$system32$\bsl$cmd.exe}
}
}
\T{Resource Monitor}{
Another tool from the System Configuration panel we will resport to now is the Resource Monitor, available under the abbreviation \cd{resmon} and executable as \cd{resmon.exe}.\\
It displays per-provess and aggregate usage information of all relevant resources on the system, as well as access and usage information on the file system by the applications and processes.\\
It allows filtering to reduec the displayed information, as well as individual actions on the eventually problematic applications.\\
For each of the main resources (CPU, Disk, Network and Memory) it displays a tab with the most relevant information and the corresponding process id (PID).\\
Further detailed information on each of the resources is accessible from the corresponding tab at the top, together with a graphical view of the real time consumption for each relevant category.
\QA{
Under Environment Variables, what is the value for ComSpec? 
}{
An informative task as it is, we just read the executable form above: \cd{resmon.exe}
}
}
\T{Command Prompt}{
The next tool of the System Configuration is the command prompt, callable via \cd{cmd.exe}.\\
This is the way to run commands, similar to the Terminal in Linux, directly in the system.\\
E.g. one can run the classic \cd{hostname, whoami} to see the names of the system and user respectively, and especially the command \cd{ipconfig} is useful to see the network address settings. In order to specify switches as one would do in Linux using \cd{-s} or \cd{--switch}, in Windows we use \cd{/switch}. E.g, the ``help'' switch is called with \cd{/?}.\\
To clear the command prompt screen we use the command \cd{cls}.\\
We futher have the command \cd{netstat} at our disposal to see the protocol statistics and current TCP/IP network connections. This command has switches \cd{-a, -b, -e} in the more traditional fashion to alter the command output. \\
The command \cd{net} is used (in mandatory combination with subcommands) to manage network resources. If no subcommand is provided, a basic syntax helper is displayed.\\
For further syntax help we better use \cd{net help <subcommand>}, as \cd{net /?} wont work here.\\
Some nice applications of this command are the listing of all users with further specification granularity options.\\
\QA{
In System Configuration, what is the full command for Internet Protocol Configuration?
}{
As we already did so often before, we can read the command plainly in System Configuration: \cd{C:$\bsl$Windows$\bsl$System32$\bsl$cmd.exe /k \%windir\%$\bsl$system32$\bsl$ipconfig.exe}
}
\QA{
For the ipconfig command, how do you show detailed information?
}{
Using the above information we run \cd{ipconfig /?} and see that \cd{/all} ``Display[s] full configuration information'', hence giving the full command \cd{ipconfig /all}
}
}
\T{Registry Editor}{
The last tool of the System COnfiguration Panel we will deal with in detail in this section is the Windows Registry.\\
It is a ``central hierarchical database used to store information necessary  to configure the system for one or more users, applications and hardware devices''.\\
Here the information Windows constantly needs to reference is stored for faster access, saving user profiles, applications and their related document creation permissions, property sheet settings for folders and applications, peripherical hardware information and port information.\\
It is considered to be an advanced tool, not intended for the standard user.\\
It is accessible by running \cd{regedit} or \cd{regedt32.exe}, as we see in the System Configuration.\\
\QA{
What is the command to open the Registry Editor? (The answer is the name of  the .exe file, not the full path) 
}{
\cd{regedt32.exe}
}
}