\subsubsection*{Windows Fundamentals 1}
\addcontentsline{toc}{subsubsection}{Windows Fundamentals 1}
\T{Introduction to Windows}{
This module will cover some Windows basics.\\
We first lauch and deploy the Windows machine corresponding to this module.
}
\T{Windows Editions}{
Windows first started on 1985 and is the most widespread OS in domestic and business environments, being consequently targeted as entry vector by attackers.\\
After a short review of the history of Windows versions, Windows 10 is presented as the current most common OS:\\
There are two Windows versions: Home for domestic use and Pro usually for companies. \\

The most common OS for Windows servers is Windows Server 2019.\\

Windows 10 will only be supported until October 13, 2025.\\
Reading more on the linked Windows features \href{https://www.microsoft.com/en-us/windows/compare-windows-10-home-vs-pro}{here} we can also see a comparison between the Home and Pro versions of Windows. 
\QA{
What encryption can you enable on Pro that you can't enable in Home?
}{
BitLocker
}
}
\T{The Desktop (GUI)}{
This first ``practical'' task deals with the Windows Desktop, the anchor every Windows user returns to every time after entering valid credentials to log in as said user.\\
The usual subdivision of a Windows Desktop consists of: 
\begin{enumerate}
\item \textbf{Desktop}: It is per se the main area, where all shortcuts to programs, folders and other locations of the file system are located. It can be personalized under Display Settings. 
\item \textbf{Start Menu}: Given by the Windows Logo on the bottom left corner of the Desktop GUI, it provides access to all programs and files, sorted by relevance and/or alphabetical order. \\
Here the user can also access their settings and an alphabetical grid to find programs by initial letter. 
\item \textbf{Taskbar}: On the lower edge of the Desktop GUI we have a Taskbar with or without a search assistant (Cortana), a toolbar and some pinned programs, if the user has previously done so.\\
Any open programs will also show here. 
\item \textbf{Notification Area}: Located at the bottom right corner, it shows the date and time, as well as eventual notifications. 
\end{enumerate}
\QA{
Which selection will hide/disable the Search Box?
}{
Hidden
}
\QA{
Which selection will hide/disable the Task View button?
}{
Show Task View Button
}
\QA{
Besides Clock and Network, what other icon is visible in the notification area?
}{
Action Center
}
}
\T{The File System}{
Windows uses NTFS (New Technology File System) as file system. Before NTFS, FAT16/FAT32 (File Allocation Table) and HPFS (High performance File System) were the file systems in place, not completely overhauled yet, especially FAT partitions in portable devices such as USB and MicroSD drives, but not any more on personal laptops and Windows servers.\\
NTFS is a journaling file system, meaning that it can repair folders and files calling their information from a log file. This is a clear improvement with respect to FAT.\\
Further advantages of NTFS are great size management, as it can support even more than 4GB, tailored permission settings for files and folders, compression of data and encryption using EFS (Encryption File System).\\
One can check which file system the current installation is running under the Properties tab of the drive in question\\

NTFS grants the following permission possibilities:
\begin{itemize}
\item Full Control
\item Modify
\item Read \& Execute
\item List folder contents
\item Read
\item Write
\end{itemize}
To view said permissions for a file or folder, one shall right-click on the piece of data the user is interested in, and there go to the tab Properties > Security, and under Group or user names one can choose the object of their interest, seeing the corresponding permissions under the field ``Permissions for Users''.\\

NTFS also implements ADS (Alternate Data Streams), a file attribute allowing for a file to contain more than one stream of data (\$DATA). To view the ADS for files one has to use the Powershell or 3rd-party software, as the standard Windows Explorer doesn't allow to view it. It is a common place for malware writers to hide data. 
\QA{
What is the meaning of NTFS?
}{
New Technology File System
}
}
\T{The Windows$\backslash$System32 Folders}{
The Windows OS is usually located on the main disk under \cd{C:$\bsl$Windows}, although the location within the C drive is not mandatory.\\
In order to circumvent the possible different locations of this folder, one uses environment variables, defined as ``"Environment variables store information about the operating system environment. This information includes details such as the operating system path, the number of processors used by the operating system, and the location of temporary folders''.\\
The one for the Windows directory is \cd{\%windir\%}.\\

Within the \cd{Windows} folder, there is a myriad of folders, among which \cd{System32} holds especial relevance, as any deletion within would render the OS inoperational. 
\QA{
What is the system variable for the Windows folder?
}{
\%windir\%
}
}
\T{User Accounts, Profiles and Permissions}{
There are two types of user accounts on a standard Windows system: Administrator and Standard. Depending on the user account status some actions are allowed: any user can change data attributed to them, but only administrators can install programs, add or delete users and groups, modify setting or perform any other system-level changes. \\
On the deployed machine, our user has Administrator rights. In order to determine the existing accounts we search for ``Other Users'' in the Start Menu, seeing the shortcut to the corresponding tab in the System Settings.\\
As an Administrator, one can ``Add someone else to this PC'', also Change the account type or remove existing users.\\
When creating user accounts, a profile is created upon initial login and the user's data are stored under \cd{C:$\bsl$Users$\bsl$<Username>}. Common directories for all users are: Desktop, Documents, Downloads, Music and Pictures.\\

Under the Local User and Group Management one can access all the information related to said users and groups. We access it via right-clicking on the start menu and on ``Run'', then running \cd{lusrmgr.msc}\\
Here we can see all Users and groups. Whenever we assign a user to a group, it inherits all group permissions. 
\QA{
What is the name of the other user account?
}{
tryhackmebilly
}
\QA{
What groups is this user a member of?
}{
Users, Remote Desktop Users
}
\QA{
What built-in account is for guest access to the computer?
}{
Guests
}
\QA{
What is the account status?
}{
Account is disabled
}
}
\T{User Account Control}{
Most home users are logged as local administrators, which poses an inherent security risk when system alteration is not prevented otherwise, in Windows via User Account Control (UAC) for all users but the ``actual'' administrator account.\\
UAC lets any user run their session with non elevated permissions and asks for authentication whenever an execute action is attempted. We can see the execute permissions for every file, program or shortcut under ``Properties'' after right-clicking on the relevant data.\\
As a visual aid, whenever authentication is further needed, a shield icon is displayed on the non authorized piece of data.\\
\QA{
What does UAC mean?
}{
User Account Control
}
}
\T{Settings and the Control Panel}{
On the Windows OS, the Settings Menu and the Control Panel are the places to go to adjust the settings: the Control Planel in charge of more complex system changes and the Settings Menu for (nowadays) most of the changes as a primary location.\\
Some times, the basic settings of the Settings Menu will take a user to the Control Panel whenever more advanced adjustments are needed. 
\QA{
In the Control Panel, change the view to Small icons. What is the last setting in the Control Panel view?
}{
Windows Defender Firewall
}
}
\T{Task Manager}{
The Task Manager provides inforamtion about the current state of the system such as running processes and applications as well as resources drained by them.\\
One can open the Task Manager by right-clicking the taskbar or with the shortcut ``Ctrl+Shift+Esc''.\\
Under ``More details'' one can then see the actual state of the system. 
\QA{
What is the keyboard shortcut to open the Task Manager? 
}{
Ctrl+Shift+Esc
}
}